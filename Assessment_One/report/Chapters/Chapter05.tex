%************************************************
\chapter{Testing report}\label{ch:testing_report} % $\mathbb{ZNR}$
%************************************************

The application is delivered with a set of test cases that all pass: they are printed in \autoref{ch:appendix}\footnote{In order to run the test cases it is necessary that the \href{http://www.junit.org/}{JUnit} (\url{http://www.junit.org/}) and the \href{http://code.google.com/p/mockito/}{Mockito} (\url{http://code.google.com/p/mockito/}) framework are present.}.

A short list of noteworthy test cases shall be provided as an overview:

\begin{description}
\item[File handling:] 

\begin{itemize}
\item Create new Cabin from valid input string.
\item Create new cabin from invalid input string (throws IllegalArgumentException).
\item Read from non present file (catch with stack trace).
\item Write to non present file (catch with stack trace).
\end{itemize}
\item[Cabin:]

\begin{itemize}
\item Set number of beds (too few $\rightarrow$ IllegalArgumentException, too many $\rightarrow$ IllegalArgumentException, valid numbers).
\item Test cost calculation algorithm.
\end{itemize}
\item[Cabin manager:]

\begin{itemize}
\item Add cabin.
\item Get cabin at index (with and without any cabins $\rightarrow$ IndexOutOfBoundsException).
\item Find cabin by number (cabin is in list and cabin is not in list $\rightarrow$ CabinNotFoundException).
\item Get cheapest and most expensive cabin (without cabins $\rightarrow$ NoCabinsException).
\end{itemize}
\end{description}

The error handling of incorrect input in the file handler is performed by checking each attribute and - if an error occurs - add this attribute to an error-string that will be returned via an IllegalArgumentException. This way it is possible to inform the user about the concrete argument that caused the error.